% texinfo.tex -- TeX macros to handle Texinfo files.
%
% Load plain if necessary, i.e., if running under initex.
\expandafter\ifx\csname fmtname\endcsname\relax\input plain\fi
%
\def\texinfoversion{2002-06-04.06}
%
% Copyright (C) 1985, 86, 88, 90, 91, 92, 93, 94, 95, 96, 97, 98, 99,
%               2000, 01, 02 Free Software Foundation, Inc.
%
% This texinfo.tex file is free software; you can redistribute it and/or
% modify it under the terms of the GNU General Public License as
% published by the Free Software Foundation; either version 2, or (at
% your option) any later version.
%
% This texinfo.tex file is distributed in the hope that it will be
% useful, but WITHOUT ANY WARRANTY; without even the implied warranty
% of MERCHANTABILITY or FITNESS FOR A PARTICULAR PURPOSE.  See the GNU
% General Public License for more details.
%
% You should have received a copy of the GNU General Public License
% along with this texinfo.tex file; see the file COPYING.  If not, write
% to the Free Software Foundation, Inc., 59 Temple Place - Suite 330,
% Boston, MA 02111-1307, USA.
%
% In other words, you are welcome to use, share and improve this program.
% You are forbidden to forbid anyone else to use, share and improve
% what you give them.   Help stamp out software-hoarding!
%
% Please try the latest version of texinfo.tex before submitting bug
% reports; you can get the latest version from:
%   ftp://ftp.gnu.org/gnu/texinfo.tex
%     (and all GNU mirrors, see http://www.gnu.org/order/ftp.html)
%   ftp://texinfo.org/texinfo/texinfo.tex
%   ftp://tug.org/tex/texinfo.tex
%     (and all CTAN mirrors, see http://www.ctan.org),
%   and /home/gd/gnu/doc/texinfo.tex on the GNU machines.
% 
% The texinfo.tex in any given Texinfo distribution could well be out
% of date, so if that's what you're using, please check.
% 
% Texinfo has a small home page at http://texinfo.org/ and also
% http://www.gnu.org/software/texinfo.
%
% Send bug reports to bug-texinfo@gnu.org.  Please include including a
% complete document in each bug report with which we can reproduce the
% problem.  Patches are, of course, greatly appreciated.
%
% To process a Texinfo manual with TeX, it's most reliable to use the
% texi2dvi shell script that comes with the distribution.  For a simple
% manual foo.texi, however, you can get away with this:
%   tex foo.texi
%   texindex foo.??
%   tex foo.texi
%   tex foo.texi
%   dvips foo.dvi -o  # or whatever; this makes foo.ps.
% The extra TeX runs get the cross-reference information correct.
% Sometimes one run after texindex suffices, and sometimes you need more
% than two; texi2dvi does it as many times as necessary.
%
% It is possible to adapt texinfo.tex for other languages.  You can get
% the existing language-specific files from the full Texinfo distribution.

\message{Loading texinfo [version \texinfoversion]:}

% If in a .fmt file, print the version number
% and turn on active characters that we couldn't do earlier because
% they might have appeared in the input file name.
\everyjob{\message{[Texinfo version \texinfoversion]}%
  \catcode`+=\active \catcode`\_=\active}

% Save some parts of plain tex whose names we will redefine.
\let\ptexb=\b
\let\ptexbullet=\bullet
\let\ptexc=\c
\let\ptexcomma=\,
\let\ptexdot=\.
\let\ptexdots=\dots
\let\ptexend=\end
\let\ptexequiv=\equiv
\let\ptexexclam=\!
\let\ptexi=\i
\let\ptexlbrace=\{
\let\ptexrbrace=\}
\let\ptexstar=\*
\let\ptext=\t

% We never want plain's outer \+ definition in Texinfo.
% For @tex, we can use \tabalign.
\let\+ = \relax

\message{Basics,}
\chardef\other=12

% If this character appears in an error message or help string, it
% starts a new line in the output.
\newlinechar = `^^J

% Set up fixed words for English if not already set.
\ifx\putwordAppendix\undefined  \gdef\putwordAppendix{Appendix}\fi
\ifx\putwordChapter\undefined   \gdef\putwordChapter{Chapter}\fi
\ifx\putwordfile\undefined      \gdef\putwordfile{file}\fi
\ifx\putwordin\undefined        \gdef\putwordin{in}\fi
\ifx\putwordIndexIsEmpty\undefined     \gdef\putwordIndexIsEmpty{(Index is empty)}\fi
\ifx\putwordIndexNonexistent\undefined \gdef\putwordIndexNonexistent{(Index is nonexistent)}\fi
\ifx\putwordInfo\undefined      \gdef\putwordInfo{Info}\fi
\ifx\putwordInstanceVariableof\undefined \gdef\putwordInstanceVariableof{Instance Variable of}\fi
\ifx\putwordMethodon\undefined  \gdef\putwordMethodon{Method on}\fi
\ifx\putwordNoTitle\undefined   \gdef\putwordNoTitle{No Title}\fi
\ifx\putwordof\undefined        \gdef\putwordof{of}\fi
\ifx\putwordon\undefined        \gdef\putwordon{on}\fi
\ifx\putwordpage\undefined      \gdef\putwordpage{page}\fi
\ifx\putwordsection\undefined   \gdef\putwordsection{section}\fi
\ifx\putwordSection\undefined   \gdef\putwordSection{Section}\fi
\ifx\putwordsee\undefined       \gdef\putwordsee{see}\fi
\ifx\putwordSee\undefined       \gdef\putwordSee{See}\fi
\ifx\putwordShortTOC\undefined  \gdef\putwordShortTOC{Short Contents}\fi
\ifx\putwordTOC\undefined       \gdef\putwordTOC{Table of Contents}\fi
%
\ifx\putwordMJan\undefined \gdef\putwordMJan{January}\fi
\ifx\putwordMFeb\undefined \gdef\putwordMFeb{February}\fi
\ifx\putwordMMar\undefined \gdef\putwordMMar{March}\fi
\ifx\putwordMApr\undefined \gdef\putwordMApr{April}\fi
\ifx\putwordMMay\undefined \gdef\putwordMMay{May}\fi
\ifx\putwordMJun\undefined \gdef\putwordMJun{June}\fi
\ifx\putwordMJul\undefined \gdef\putwordMJul{July}\fi
\ifx\putwordMAug\undefined \gdef\putwordMAug{August}\fi
\ifx\putwordMSep\undefined \gdef\putwordMSep{September}\fi
\ifx\putwordMOct\undefined \gdef\putwordMOct{October}\fi
\ifx\putwordMNov\undefined \gdef\putwordMNov{November}\fi
\ifx\putwordMDec\undefined \gdef\putwordMDec{December}\fi
%
\ifx\putwordDefmac\undefined    \gdef\putwordDefmac{Macro}\fi
\ifx\putwordDefspec\undefined   \gdef\putwordDefspec{Special Form}\fi
\ifx\putwordDefvar\undefined    \gdef\putwordDefvar{Variable}\fi
\ifx\putwordDefopt\undefined    \gdef\putwordDefopt{User Option}\fi
\ifx\putwordDeftypevar\undefined\gdef\putwordDeftypevar{Variable}\fi
\ifx\putwordDeffunc\undefined   \gdef\putwordDeffunc{Function}\fi
\ifx\putwordDeftypefun\undefined\gdef\putwordDeftypefun{Function}\fi

% Ignore a token.
%
\def\gobble#1{}

\hyphenation{ap-pen-dix}
\hyphenation{mini-buf-fer mini-buf-fers}
\hyphenation{eshell}
\hyphenation{white-space}

% Margin to add to right of even pages, to left of odd pages.
\newdimen \bindingoffset
\newdimen \normaloffset
\newdimen\pagewidth \newdimen\pageheight

% Sometimes it is convenient to have everything in the transcript file
% and nothing on the terminal.  We don't just call \tracingall here,
% since that produces some useless output on the terminal.
%
\def\gloggingall{\begingroup \globaldefs = 1 \loggingall \endgroup}%
\ifx\eTeXversion\undefined
\def\loggingall{\tracingcommands2 \tracingstats2
   \tracingpages1 \tracingoutput1 \tracinglostchars1
   \tracingmacros2 \tracingparagraphs1 \tracingrestores1
   \showboxbreadth\maxdimen\showboxdepth\maxdimen
}%
\else
\def\loggingall{\tracingcommands3 \tracingstats2
   \tracingpages1 \tracingoutput1 \tracinglostchars1
   \tracingmacros2 \tracingparagraphs1 \tracingrestores1
   \tracingscantokens1 \tracingassigns1 \tracingifs1
   \tracinggroups1 \tracingnesting2
   \showboxbreadth\maxdimen\showboxdepth\maxdimen
}%
\fi

% add check for \lastpenalty to plain's definitions.  If the last thing
% we did was a \nobreak, we don't want to insert more space.
% 
\def\smallbreak{\ifnum\lastpenalty<10000\par\ifdim\lastskip<\smallskipamount
  \removelastskip\penalty-50\smallskip\fi\fi}
\def\medbreak{\ifnum\lastpenalty<10000\par\ifdim\lastskip<\medskipamount
  \removelastskip\penalty-100\medskip\fi\fi}
\def\bigbreak{\ifnum\lastpenalty<10000\par\ifdim\lastskip<\bigskipamount
  \removelastskip\penalty-200\bigskip\fi\fi}

% For @cropmarks command.
% Do @cropmarks to get crop marks.
%
\newif\ifcropmarks
\let\cropmarks = \cropmarkstrue
%
% Dimensions to add cropmarks at corners.
% Added by P. A. MacKay, 12 Nov. 1986
%
\newdimen\outerhsize \newdimen\outervsize % set by the paper size routines
\newdimen\cornerlong  \cornerlong=1pc
\newdimen\cornerthick \cornerthick=.3pt
\newdimen\topandbottommargin \topandbottommargin=.75in

% Main output routine.
\chardef\PAGE = 255
\output = {\onepageout{\pagecontents\PAGE}}

\newbox\headlinebox
\newbox\footlinebox

% \onepageout takes a vbox as an argument.  Note that \pagecontents
% does insertions, but you have to call it yourself.
\def\onepageout#1{%
  \ifcropmarks \hoffset=0pt \else \hoffset=\normaloffset \fi
  %
  \ifodd\pageno  \advance\hoffset by \bindingoffset
  \else \advance\hoffset by -\bindingoffset\fi
  %
  % Do this outside of the \shipout so @code etc. will be expanded in
  % the headline as they should be, not taken literally (outputting ''code).
  \setbox\headlinebox = \vbox{\let\hsize=\pagewidth \makeheadline}%
  \setbox\footlinebox = \vbox{\let\hsize=\pagewidth \makefootline}%
  %
  {%
    % Have to do this stuff outside the \shipout because we want it to
    % take effect in \write's, yet the group defined by the \vbox ends
    % before the \shipout runs.
    %
    \escapechar = `\\     % use backslash in output files.
    \indexdummies         % don't expand commands in the output.
    \normalturnoffactive  % \ in index entries must not stay \, e.g., if
                   % the page break happens to be in the middle of an example.
    \shipout\vbox{%
      % Do this early so pdf references go to the beginning of the page.
      \ifpdfmakepagedest \pdfmkdest{\the\pageno} \fi
      %
      \ifcropmarks \vbox to \outervsize\bgroup
        \hsize = \outerhsize
        \vskip-\topandbottommargin
        \vtop to0pt{%
          \line{\ewtop\hfil\ewtop}%
          \nointerlineskip
          \line{%
            \vbox{\moveleft\cornerthick\nstop}%
            \hfill
            \vbox{\moveright\cornerthick\nstop}%
          }%
          \vss}%
        \vskip\topandbottommargin
        \line\bgroup
          \hfil % center the page within the outer (page) hsize.
          \ifodd\pageno\hskip\bindingoffset\fi
          \vbox\bgroup
      \fi
      %
      \unvbox\headlinebox
      \pagebody{#1}%
      \ifdim\ht\footlinebox > 0pt
        % Only leave this space if the footline is nonempty.
        % (We lessened \vsize for it in \oddfootingxxx.)
        % The \baselineskip=24pt in plain's \makefootline has no effect.
        \vskip 2\baselineskip
        \unvbox\footlinebox
      \fi
      %
      \ifcropmarks
          \egroup % end of \vbox\bgroup
        \hfil\egroup % end of (centering) \line\bgroup
        \vskip\topandbottommargin plus1fill minus1fill
        \boxmaxdepth = \cornerthick
        \vbox to0pt{\vss
          \line{%
            \vbox{\moveleft\cornerthick\nsbot}%
            \hfill
            \vbox{\moveright\cornerthick\nsbot}%
          }%
          \nointerlineskip
          \line{\ewbot\hfil\ewbot}%
        }%
      \egroup % \vbox from first cropmarks clause
      \fi
    }% end of \shipout\vbox
  }% end of group with \turnoffactive
  \advancepageno
  \ifnum\outputpenalty>-20000 \else\dosupereject\fi
}

\newinsert\margin \dimen\margin=\maxdimen

\def\pagebody#1{\vbox to\pageheight{\boxmaxdepth=\maxdepth #1}}
{\catcode`\@ =11
\gdef\pagecontents#1{\ifvoid\topins\else\unvbox\topins\fi
% marginal hacks, juha@viisa.uucp (Juha Takala)
\ifvoid\margin\else % marginal info is present
  \rlap{\kern\hsize\vbox to\z@{\kern1pt\box\margin \vss}}\fi
\dimen@=\dp#1 \unvbox#1
\ifvoid\footins\else\vskip\skip\footins\footnoterule \unvbox\footins\fi
\ifr@ggedbottom \kern-\dimen@ \vfil \fi}
}

% Here are the rules for the cropmarks.  Note that they are
% offset so that the space between them is truly \outerhsize or \outervsize
% (P. A. MacKay, 12 November, 1986)
%
\def\ewtop{\vrule height\cornerthick depth0pt width\cornerlong}
\def\nstop{\vbox
  {\hrule height\cornerthick depth\cornerlong width\cornerthick}}
\def\ewbot{\vrule height0pt depth\cornerthick width\cornerlong}
\def\nsbot{\vbox
  {\hrule height\cornerlong depth\cornerthick width\cornerthick}}

% Parse an argument, then pass it to #1.  The argument is the rest of
% the input line (except we remove a trailing comment).  #1 should be a
% macro which expects an ordinary undelimited TeX argument.
%
\def\parsearg#1{%
  \let\next = #1%
  \begingroup
    \obeylines
    \futurelet\temp\parseargx
}

% If the next token is an obeyed space (from an @example environment or
% the like), remove it and recurse.  Otherwise, we're done.
\def\parseargx{%
  % \obeyedspace is defined far below, after the definition of \sepspaces.
  \ifx\obeyedspace\temp
    \expandafter\parseargdiscardspace
  \else
    \expandafter\parseargline
  \fi
}

% Remove a single space (as the delimiter token to the macro call).
{\obeyspaces %
 \gdef\parseargdiscardspace {\futurelet\temp\parseargx}}

{\obeylines %
  \gdef\parseargline#1^^M{%
    \endgroup % End of the group started in \parsearg.
    %
    % First remove any @c comment, then any @comment.
    % Result of each macro is put in \toks0.
    \argremovec #1\c\relax %
    \expandafter\argremovecomment \the\toks0 \comment\relax %
    %
    % Call the caller's macro, saved as \next in \parsearg.
    \expandafter\next\expandafter{\the\toks0}%
  }%
}

% Since all \c{,omment} does is throw away the argument, we can let TeX
% do that for us.  The \relax here is matched by the \relax in the call
% in \parseargline; it could be more or less anything, its purpose is
% just to delimit the argument to the \c.
\def\argremovec#1\c#2\relax{\toks0 = {#1}}
\def\argremovecomment#1\comment#2\relax{\toks0 = {#1}}

% \argremovec{,omment} might leave us with trailing spaces, though; e.g.,
%    @end itemize  @c foo
% will have two active spaces as part of the argument with the
% `itemize'.  Here we remove all active spaces from #1, and assign the
% result to \toks0.
%
% This loses if there are any *other* active characters besides spaces
% in the argument -- _ ^ +, for example -- since they get expanded.
% Fortunately, Texinfo does not define any such commands.  (If it ever
% does, the catcode of the characters in questionwill have to be changed
% here.)  But this means we cannot call \removeactivespaces as part of
% \argremovec{,omment}, since @c uses \parsearg, and thus the argument
% that \parsearg gets might well have any character at all in it.
%
\def\removeactivespaces#1{%
  \begingroup
    \ignoreactivespaces
    \edef\temp{#1}%
    \global\toks0 = \expandafter{\temp}%
  \endgroup
}

% Change the active space to expand to nothing.
%
\begingroup
  \obeyspaces
  \gdef\ignoreactivespaces{\obeyspaces\let =\empty}
\endgroup


\def\flushcr{\ifx\par\lisppar \def\next##1{}\else \let\next=\relax \fi \next}

%% These are used to keep @begin/@end levels from running away
%% Call \inENV within environments (after a \begingroup)
\newif\ifENV \ENVfalse \def\inENV{\ifENV\relax\else\ENVtrue\fi}
\def\ENVcheck{%
\ifENV\errmessage{Still within an environment; press RETURN to continue}
\endgroup\fi} % This is not perfect, but it should reduce lossage

% @begin foo  is the same as @foo, for now.
\newhelp\EMsimple{Press RETURN to continue.}

\outer\def\begin{\parsearg\beginxxx}

\def\beginxxx #1{%
\expandafter\ifx\csname #1\endcsname\relax
{\errhelp=\EMsimple \errmessage{Undefined command @begin #1}}\else
\csname #1\endcsname\fi}

% @end foo executes the definition of \Efoo.
%
\def\end{\parsearg\endxxx}
\def\endxxx #1{%
  \removeactivespaces{#1}%
  \edef\endthing{\the\toks0}%
  %
  \expandafter\ifx\csname E\endthing\endcsname\relax
    \expandafter\ifx\csname \endthing\endcsname\relax
      % There's no \foo, i.e., no ``environment'' foo.
      \errhelp = \EMsimple
      \errmessage{Undefined command `@end \endthing'}%
    \else
      \unmatchedenderror\endthing
    \fi
  \else
    % Everything's ok; the right environment has been started.
    \csname E\endthing\endcsname
  \fi
}

% There is an environment #1, but it hasn't been started.  Give an error.
%
\def\unmatchedenderror#1{%
  \errhelp = \EMsimple
  \errmessage{This `@end #1' doesn't have a matching `@#1'}%
}

% Define the control sequence \E#1 to give an unmatched @end error.
%
\def\defineunmatchedend#1{%
  \expandafter\def\csname E#1\endcsname{\unmatchedenderror{#1}}%
}


% Single-spacing is done by various environments (specifically, in
% \nonfillstart and \quotations).
\newskip\singlespaceskip \singlespaceskip = 12.5pt
\def\singlespace{%
  % Why was this kern here?  It messes up equalizing space above and below
  % environments.  --karl, 6may93
  %{\advance \baselineskip by -\singlespaceskip
  %\kern \baselineskip}%
  \setleading\singlespaceskip
}

%% Simple single-character @ commands

% @@ prints an @
% Kludge this until the fonts are right (grr).
\def\@{{\tt\char64}}

% This is turned off because it was never documented
% and you can use @w{...} around a quote to suppress ligatures.
%% Define @` and @' to be the same as ` and '
%% but suppressing ligatures.
%\def\`{{`}}
%\def\'{{'}}

% Used to generate quoted braces.
\def\mylbrace {{\tt\char123}}
\def\myrbrace {{\tt\char125}}
\let\{=\mylbrace
\let\}=\myrbrace
\begingroup
  % Definitions to produce actual \{ & \} command in an index.
  \catcode`\{ = 12 \catcode`\} = 12
  \catcode`\[ = 1 \catcode`\] = 2
  \catcode`\@ = 0 \catcode`\\ = 12
  @gdef@lbracecmd[\{]%
  @gdef@rbracecmd[\}]%
@endgroup

% Accents: @, @dotaccent @ringaccent @ubaraccent @udotaccent
% Others are defined by plain TeX: @` @' @" @^ @~ @= @v @H.
\let\, = \c
\let\dotaccent = \.
\def\ringaccent#1{{\accent23 #1}}
\let\tieaccent = \t
\let\ubaraccent = \b
\let\udotaccent = \d

% Other special characters: @questiondown @exclamdown
% Plain TeX defines: @AA @AE @O @OE @L (and lowercase versions) @ss.
\def\questiondown{?`}
\def\exclamdown{!`}

% Dotless i and dotless j, used for accents.
\def\imacro{i}
\def\jmacro{j}
\def\dotless#1{%
  \def\temp{#1}%
  \ifx\temp\imacro \ptexi
  \else\ifx\temp\jmacro \j
  \else \errmessage{@dotless can be used only with i or j}%
  \fi\fi
}

% Be sure we're in horizontal mode when doing a tie, since we make space
% equivalent to this in @example-like environments. Otherwise, a space
% at the beginning of a line will start with \penalty -- and
% since \penalty is valid in vertical mode, we'd end up putting the
% penalty on the vertical list instead of in the new paragraph.
{\catcode`@ = 11
 % Avoid using \@M directly, because that causes trouble
 % if the definition is written into an index file.
 \global\let\tiepenalty = \@M
 \gdef\tie{\leavevmode\penalty\tiepenalty\ }
}

% @: forces normal size whitespace following.
\def\:{\spacefactor=1000 }

% @* forces a line break.
\def\*{\hfil\break\hbox{}\ignorespaces}

% @. is an end-of-sentence period.
\def\.{.\spacefactor=3000 }

% @! is an end-of-sentence bang.
\def\!{!\spacefactor=3000 }

% @? is an end-of-sentence query.
\def\?{?\spacefactor=3000 }

% @w prevents a word break.  Without the \leavevmode, @w at the
% beginning of a paragraph, when TeX is still in vertical mode, would
% produce a whole line of output instead of starting the paragraph.
\def\w#1{\leavevmode\hbox{#1}}

% @group ... @end group forces ... to be all on one page, by enclosing
% it in a TeX vbox.  We use \vtop instead of \vbox to construct the box
% to keep its height that of a normal line.  According to the rules for
% \topskip (p.114 of the TeXbook), the glue inserted is
% max (\topskip - \ht (first item), 0).  If that height is large,
% therefore, no glue is inserted, and the space between the headline and
% the text is small, which looks bad.
%
\def\group{\begingroup
  \ifnum\catcode13=\active \else
    \errhelp = \groupinvalidhelp
    \errmessage{@group invalid in context where filling is enabled}%
  \fi
  %
  % The \vtop we start below produces a box with normal height and large
  % depth; thus, TeX puts \baselineskip glue before it, and (when the
  % next line of text is done) \lineskip glue after it.  (See p.82 of
  % the TeXbook.)  Thus, space below is not quite equal to space
  % above.  But it's pretty close.
  \def\Egroup{%
    \egroup           % End the \vtop.
    \endgroup         % End the \group.
  }%
  %
  \vtop\bgroup
    % We have to put a strut on the last line in case the @group is in
    % the midst of an example, rather than completely enclosing it.
    % Otherwise, the interline space between the last line of the group
    % and the first line afterwards is too small.  But we can't put the
    % strut in \Egroup, since there it would be on a line by itself.
    % Hence this just inserts a strut at the beginning of each line.
    \everypar = {\strut}%
    %
    % Since we have a strut on every line, we don't need any of TeX's
    % normal interline spacing.
    \offinterlineskip
    %
    % OK, but now we have to do something about blank
    % lines in the input in @example-like environments, which normally
    % just turn into \lisppar, which will insert no space now that we've
    % turned off the interline space.  Simplest is to make them be an
    % empty paragraph.
    \ifx\par\lisppar
      \edef\par{\leavevmode \par}%
      %
      % Reset ^^M's definition to new definition of \par.
      \obeylines
    \fi
    %
    % Do @comment since we are called inside an environment such as
    % @example, where each end-of-line in the input causes an
    % end-of-line in the output.  We don't want the end-of-line after
    % the `@group' to put extra space in the output.  Since @group
    % should appear on a line by itself (according to the Texinfo
    % manual), we don't worry about eating any user text.
    \comment
}
%
% TeX puts in an \escapechar (i.e., `@') at the beginning of the help
% message, so this ends up printing `@group can only ...'.
%
\newhelp\groupinvalidhelp{%
group can only be used in environments such as @example,^^J%
where each line of input produces a line of output.}

% @need space-in-mils
% forces a page break if there is not space-in-mils remaining.

\newdimen\mil  \mil=0.001in

\def\need{\parsearg\needx}

% Old definition--didn't work.
%\def\needx #1{\par %
%% This method tries to make TeX break the page naturally
%% if the depth of the box does not fit.
%{\baselineskip=0pt%
%\vtop to #1\mil{\vfil}\kern -#1\mil\nobreak
%\prevdepth=-1000pt
%}}

\def\needx#1{%
  % Ensure vertical mode, so we don't make a big box in the middle of a
  % paragraph.
  \par
  %
  % If the @need value is less than one line space, it's useless.
  \dimen0 = #1\mil
  \dimen2 = \ht\strutbox
  \advance\dimen2 by \dp\strutbox
  \ifdim\dimen0 > \dimen2
    %
    % Do a \strut just to make the height of this box be normal, so the
    % normal leading is inserted relative to the preceding line.
    % And a page break here is fine.
    \vtop to #1\mil{\strut\vfil}%
    %
    % TeX does not even consider page breaks if a penalty added to the
    % main vertical list is 10000 or more.  But in order to see if the
    % empty box we just added fits on the page, we must make it consider
    % page breaks.  On the other hand, we don't want to actually break the
    % page after the empty box.  So we use a penalty of 9999.
    %
    % There is an extremely small chance that TeX will actually break the
    % page at this \penalty, if there are no other feasible breakpoints in
    % sight.  (If the user is using lots of big @group commands, which
    % almost-but-not-quite fill up a page, TeX will have a hard time doing
    % good page breaking, for example.)  However, I could not construct an
    % example where a page broke at this \penalty; if it happens in a real
    % document, then we can reconsider our strategy.
    \penalty9999
    %
    % Back up by the size of the box, whether we did a page break or not.
    \kern -#1\mil
    %
    % Do not allow a page break right after this kern.
    \nobreak
  \fi
}

% @br   forces paragraph break

\let\br = \par

% @dots{} output an ellipsis using the current font.
% We do .5em per period so that it has the same spacing in a typewriter
% font as three actual period characters.
%
\def\dots{%
  \leavevmode
  \hbox to 1.5em{%
    \hskip 0pt plus 0.25fil minus 0.25fil
    .\hss.\hss.%
    \hskip 0pt plus 0.5fil minus 0.5fil
  }%
}

% @enddots{} is an end-of-sentence ellipsis.
%
\def\enddots{%
  \leavevmode
  \hbox to 2em{%
    \hskip 0pt plus 0.25fil minus 0.25fil
    .\hss.\hss.\hss.%
    \hskip 0pt plus 0.5fil minus 0.5fil
  }%
  \spacefactor=3000
}


% @page    forces the start of a new page
%
\def\page{\par\vfill\supereject}

% @exdent text....
% outputs text on separate line in roman font, starting at standard page margin

% This records the amount of indent in the innermost environment.
% That's how much \exdent should take out.
\newskip\exdentamount

% This defn is used inside fill environments such as @defun.
\def\exdent{\parsearg\exdentyyy}
\def\exdentyyy #1{{\hfil\break\hbox{\kern -\exdentamount{\rm#1}}\hfil\break}}

% This defn is used inside nofill environments such as @example.
\def\nofillexdent{\parsearg\nofillexdentyyy}
\def\nofillexdentyyy #1{{\advance \leftskip by -\exdentamount
\leftline{\hskip\leftskip{\rm#1}}}}

% @inmargin{WHICH}{TEXT} puts TEXT in the WHICH margin next to the current
% paragraph.  For more general purposes, use the \margin insertion
% class.  WHICH is `l' or `r'.
%
\newskip\inmarginspacing \inmarginspacing=1cm
\def\strutdepth{\dp\strutbox}
%
\def\doinmargin#1#2{\strut\vadjust{%
  \nobreak
  \kern-\strutdepth
  \vtop to \strutdepth{%
    \baselineskip=\strutdepth
    \vss
    % if you have multiple lines of stuff to put here, you'll need to
    % make the vbox yourself of the appropriate size.
    \ifx#1l%
      \llap{\ignorespaces #2\hskip\inmarginspacing}%
    \else
      \rlap{\hskip\hsize \hskip\inmarginspacing \ignorespaces #2}%
    \fi
    \null
  }%
}}
\def\inleftmargin{\doinmargin l}
\def\inrightmargin{\doinmargin r}
%
% @inmargin{TEXT [, RIGHT-TEXT]}
% (if RIGHT-TEXT is given, use TEXT for left page, RIGHT-TEXT for right;
% else use TEXT for both).
% 
\def\inmargin#1{\parseinmargin #1,,\finish}
\def\parseinmargin#1,#2,#3\finish{% not perfect, but better than nothing.
  \setbox0 = \hbox{\ignorespaces #2}% 
  \ifdim\wd0 > 0pt
    \def\lefttext{#1}%  have both texts
    \def\righttext{#2}%
  \else
    \def\lefttext{#1}%  have only one text
    \def\righttext{#1}%
  \fi
  %
  \ifodd\pageno
    \def\temp{\inrightmargin\righttext}% odd page -> outside is right margin
  \else
    \def\temp{\inleftmargin\lefttext}%
  \fi
  \temp
}

% @include file    insert text of that file as input.
% Allow normal characters that  we make active in the argument (a file name).
\def\include{\begingroup
  \catcode`\\=12
  \catcode`~=12
  \catcode`^=12
  \catcode`_=12
  \catcode`|=12
  \catcode`<=12
  \catcode`>=12
  \catcode`+=12
  \parsearg\includezzz}
% Restore active chars for included file.
\def\includezzz#1{\endgroup\begingroup
  % Read the included file in a group so nested @include's work.
  \def\thisfile{#1}%
  \input\thisfile
\endgroup}

\def\thisfile{}

% @center line   outputs that line, centered

\def\center{\parsearg\centerzzz}
\def\centerzzz #1{{\advance\hsize by -\leftskip
\advance\hsize by -\rightskip
\centerline{#1}}}

% @sp n   outputs n lines of vertical space

\def\sp{\parsearg\spxxx}
\def\spxxx #1{\vskip #1\baselineskip}

% @comment ...line which is ignored...
% @c is the same as @comment
% @ignore ... @end ignore  is another way to write a comment

\def\comment{\begingroup \catcode`\^^M=\other%
\catcode`\@=\other \catcode`\{=\other \catcode`\}=\other%
\commentxxx}
{\catcode`\^^M=\other \gdef\commentxxx#1^^M{\endgroup}}

\let\c=\comment

% @paragraphindent NCHARS
% We'll use ems for NCHARS, close enough.
% We cannot implement @paragraphindent asis, though.
% 
\def\asisword{asis} % no translation, these are keywords
\def\noneword{none}
%
\def\paragraphindent{\parsearg\doparagraphindent}
\def\doparagraphindent#1{%
  \def\temp{#1}%
  \ifx\temp\asisword
  \else
    \ifx\temp\noneword
      \defaultparindent = 0pt
    \else
      \defaultparindent = #1em
    \fi
  \fi
  \parindent = \defaultparindent
}

% @exampleindent NCHARS
% We'll use ems for NCHARS like @paragraphindent.
% It seems @exampleindent asis isn't necessary, but
% I preserve it to make it similar to @paragraphindent.
\def\exampleindent{\parsearg\doexampleindent}
\def\doexampleindent#1{%
  \def\temp{#1}%
  \ifx\temp\asisword
  \else
    \ifx\temp\noneword
      \lispnarrowing = 0pt
    \else
      \lispnarrowing = #1em
    \fi
  \fi
}

% @asis just yields its argument.  Used with @table, for example.
%
\def\asis#1{#1}

% @math outputs its argument in math mode.
% We don't use $'s directly in the definition of \math because we need
% to set catcodes according to plain TeX first, to allow for subscripts,
% superscripts, special math chars, etc.
% 
% @math does not do math typesetting in section titles, index
% entries, and other such contexts where the catcodes are set before
% @math gets a chance to work.  This could perhaps be fixed, but for now
% at least we can have real math in the main text, where it's needed most.
%
\let\implicitmath = $%$ font-lock fix
%
% One complication: _ usually means subscripts, but it could also mean
% an actual _ character, as in @math{@var{some_variable} + 1}.  So make
% _ within @math be active (mathcode "8000), and distinguish by seeing
% if the current family is \slfam, which is what @var uses.
% 
{\catcode95 = \active  % 95 = _
\gdef\mathunderscore{%
  \catcode95=\active
  \def_{\ifnum\fam=\slfam \_\else\sb\fi}%
}}
%
% Another complication: we want \\ (and @\) to output a \ character.
% FYI, plain.tex uses \\ as a temporary control sequence (why?), but
% this is not advertised and we don't care.  Texinfo does not
% otherwise define @\.
% 
% The \mathchar is class=0=ordinary, family=7=ttfam, position=5C=\.
\def\mathbackslash{\ifnum\fam=\ttfam \mathchar"075C \else\backslash \fi}
%
\def\math{%
  \tex
  \mathcode`\_="8000 \mathunderscore
  \let\\ = \mathbackslash
  \implicitmath\finishmath}
\def\finishmath#1{#1\implicitmath\Etex}

% @bullet and @minus need the same treatment as @math, just above.
\def\bullet{\implicitmath\ptexbullet\implicitmath}
\def\minus{\implicitmath-\implicitmath}

% @refill is a no-op.
\let\refill=\relax

% If working on a large document in chapters, it is convenient to
% be able to disable indexing, cross-referencing, and contents, for test runs.
% This is done with @novalidate (before @setfilename).
%
\newif\iflinks \linkstrue % by default we want the aux files.
\let\novalidate = \linksfalse

% @setfilename is done at the beginning of every texinfo file.
% So open here the files we need to have open while reading the input.
% This makes it possible to make a .fmt file for texinfo.
\def\setfilename{%
   \iflinks
     \readauxfile
   \fi % \openindices needs to do some work in any case.
   \openindices
   \fixbackslash  % Turn off hack to swallow `\input texinfo'.
   \global\let\setfilename=\comment % Ignore extra @setfilename cmds.
   %
   % If texinfo.cnf is present on the system, read it.
   % Useful for site-wide @afourpaper, etc.
   % Just to be on the safe side, close the input stream before the \input.
   \openin 1 texinfo.cnf
   \ifeof1 \let\temp=\relax \else \def\temp{\input texinfo.cnf }\fi
   \closein1
   \temp
   %
   \comment % Ignore the actual filename.
}

% Called from \setfilename.
%
\def\openindices{%
  \newindex{cp}%
  \newcodeindex{fn}%
  \newcodeindex{vr}%
  \newcodeindex{tp}%
  \newcodeindex{ky}%
  \newcodeindex{pg}%
}

% @bye.
\outer\def\bye{\pagealignmacro\tracingstats=1\ptexend}


\message{pdf,}
% adobe `portable' document format
\newcount\tempnum
\newcount\lnkcount
\newtoks\filename
\newcount\filenamelength
\newcount\pgn
\newtoks\toksA
\newtoks\toksB
\newtoks\toksC
\newtoks\toksD
\newbox\boxA
\newcount\countA
\newif\ifpdf
\newif\ifpdfmakepagedest

\ifx\pdfoutput\undefined
  \pdffalse
  \let\pdfmkdest = \gobble
  \let\pdfurl = \gobble
  \let\endlink = \relax
  \let\linkcolor = \relax
  \let\pdfmakeoutlines = \relax
\else
  \pdftrue
  \pdfoutput = 1
  \input pdfcolor
  \def\dopdfimage#1#2#3{%
    \def\imagewidth{#2}%
    \def\imageheight{#3}%
    % without \immediate, pdftex seg faults when the same image is
    % included twice.  (Version 3.14159-pre-1.0-unofficial-20010704.)
    \ifnum\pdftexversion < 14
      \immediate\pdfimage
    \else
      \immediate\pdfximage
    \fi
      \ifx\empty\imagewidth\else width \imagewidth \fi
      \ifx\empty\imageheight\else height \imageheight \fi
      \ifnum\pdftexversion<13
	 #1.pdf%
       \else
         {#1.pdf}%
       \fi
    \ifnum\pdftexversion < 14 \else
      \pdfrefximage \pdflastximage
    \fi}
  \def\pdfmkdest#1{{\normalturnoffactive \pdfdest name{#1} xyz}}
  \def\pdfmkpgn#1{#1}
  \let\linkcolor = \Blue  % was Cyan, but th